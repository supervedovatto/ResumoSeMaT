%
%  Modelo com instru��es para prepara��o do resumo para submiss�o de trabalho
%  para apresenta��o em Sess�o T�cnica no XXVI CNMAC
%
%

\documentclass[a4,10pt]{article}
\usepackage{amsfonts, amsmath}
\usepackage[portuguese]{babel}
\usepackage[latin1]{inputenc}
\usepackage{fancyhdr, graphicx, epsfig}

\newcommand{\headrulecolor}[1]{\patchcmd{\headrule}{\hrule}{\color{#1}\hrule}{}{}}
\renewcommand{\headrulewidth}{0.5pt}
\fancyhead[R]{}
\fancyhead[C]{
	\begin{minipage}{3cm}
	\includegraphics[height=2cm]{logo_goiania}
	\end{minipage}
\begin{minipage}{12.5cm}
    \begin{flushright}
	\includegraphics[height=1.5cm]{logo_semat}
    \end{flushright}
\end{minipage}
}

\pagenumbering{gobble}
\headheight 20mm      %
\oddsidemargin 2.0mm  %
\evensidemargin 2.0mm %
\topmargin -25mm      %
\textheight 250mm     %
\textwidth 160mm      %

\begin{document}

\title{\Large{\bf T�tulo do Trabalho}\footnote{este trabalho conta com apoio dinanceiro de ...}}
\author{ {\bf {\large Tal, Fulano de}}\\
 {\small \{Instituto, faculdade, departamento e/ou c�mpus\},  \{SIGLA DA INSTITUI��O\}} \\
  {\small email@im.ufnt.br} \\
 {\bf {\large Silva, Ciclano Jos� da}}  \\
 {\small \{Instituto, faculdade, departamento e/ou c�mpus\},  \{SIGLA DA INSTITUI��O\}} \\
  {\small email@iffrt.edu.br}}
%\date{}

\maketitle

\thispagestyle{fancy}


%%%%%%%%%%%%%%%%%%%%%%%%%%%%%%%%%%%%%%%%%%%%%%%%%%%%%%%%%%%%%%%%%

\section{Introdu��o}

Aqui deve-voc� faz a introdu��o do trabalho.

\section{Desenvolvimento}
corpo do trabalho

\section{Considera��es Finais}
Breve considera��es sobre o trabalho

\begin{thebibliography}{99}

\bibitem{key1} BURGER, M.; HACKL, B.; RING, W. \emph{Incorporating topological derivatives
into level set methods}. Journal of Computational Physics, v. 194, n. 1, p. 344-362,
2004.


\bibitem{hig}  LITTLE, R. W. \emph{Elasticity}. New Jersey: Prentice-Hall, 1973.

\end{thebibliography}

\end{document}
